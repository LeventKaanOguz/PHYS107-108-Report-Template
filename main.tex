\documentclass[10pt]{article}
\usepackage{graphicx}
\usepackage[backend=biber,style=numeric,sorting=ynt]{biblatex}
\usepackage{appendix}
% \usepackage[margin=1in]{geometry} DON'T REMOVE THE COMMENT.
\usepackage{amsmath, amssymb}
\usepackage{hyperref}
\usepackage{bookmark}
\usepackage{titlesec}
\usepackage{fancyhdr}
\usepackage{listings}
\usepackage{xcolor}
\usepackage{float}
\usepackage{arydshln}
\usepackage{indentfirst}
\usepackage{multicol}
\usepackage{subcaption}
\usepackage{physics}
\usepackage{csvsimple}
\usepackage{lipsum} % Text filler, don't use

% For removing reference section title, we will add it manually later on.
\renewcommand{\refname}{}

\addbibresource{references.bib}

% Setup for hyperlinks, for pretty URL's.
\hypersetup{
    colorlinks=true,
    linkcolor=blue,
    filecolor=magenta,      
    urlcolor=cyan,
}

% Listings settings for MATLAB, for pretty code.
\lstset{
  language=Matlab,
  basicstyle=\footnotesize\ttfamily,
  breaklines=true,
  keywordstyle=\color{blue},
  numbers=left,
  numberstyle=\tiny\color{gray},
  stringstyle=\color{purple},
  commentstyle=\color{teal},
  morecomment=[l][\color{magenta}]{\#},
  frame=single,
  rulecolor=\color{black},
  showstringspaces=false,
  tabsize=2,
  title=\lstname % Show the filename of files included with \lstinputlisting;
}


% Title formatting, so you will not need to abuse "commands" to make it work.
\titleformat*{\section}{\large\bfseries}
\titleformat*{\subsection}{\normalsize\bfseries}
\titleformat*{\subsubsection}{\small\bfseries}

% Prevent section numbers
\setcounter{secnumdepth}{0}

\begin{document}


\begin{table}[htbp]
\begin{tabularx}{}
Name-Surname: \hfill Date: DD/MM/YYYY \\ 
Student ID:   \hfill Session:         \\ 
\end{tabularx}
\end{table}



\section{Introduction \& Background/Theory}
% Introduce both experiments and their theoretical backgrounds.
% Explain the model and its underpinnings for both experiments.

\subsection{Introduction}

\subsection{Background/Theory}


\section{Methodology}

\section{Results \& Discussion}

\section{Additional Tasks}

% Used this vspace command to reduce the distance between figures and the title, you can remove, comment or increase it as you wish.

\section{Discussion \& Conclusion}
% Combined conclusions for both experiments.



\section{Names of the Participants}


% A simple table for participants and their contributions.
\begin{table}[H]
    \centering
    \begin{tabular}{| c : c |}
    \hline
        \textbf{Participants} & \textbf{Their Contribution} \\
    \hline\hline
        John Doe (writer of this report) & Contribution of YOURSELF. \\
    \hline
        John Smith & Contribution of John Smith \\
    \hline
    \end{tabular}
    \caption{Even though everyone focused on specific things, we mostly tried each others jobs too.}
\end{table}


% In case you need a special section for Extra Credit, uncomment below line and starting doing the magic.
% \section{Extra Credit}

\printbibliography

\section{Appendices}


% Add, remove, or change the order of the items as you wish.
\begin{enumerate}
    \item MATLAB Code
    \item Images
    \item Figures % I highly recommend adding figures to the Appendices section aswell.
\end{enumerate}

\subsection{1. MATLAB Code}
% Include MATLAB code snippets or link to repository.

% Examples of how to add MATLAB code as a listing.


\subsection{2. Images}
% Include photographs, drawings, etc.
% The below command is quite buggy, it is recommended not to uncomment it.
% \setcounter{figure}{0}
% You can use figure environment to add images in a flexible way.


\subsection{3. Figures}
% Include figures

\section{Extra Credit Questions}

\end{document}
